\documentclass{article}
\usepackage[utf8]{inputenc}
\usepackage{amsmath}
\usepackage{amssymb}
\usepackage{physics}
\usepackage{gensymb}
\usepackage{esvect}
\usepackage[title]{appendix} % or other options
\usepackage{graphicx} % Required for inserting images

% Custom commands for partial derivatives
\newcommand{\pd}{\partial}  % Short for \partial
\newcommand{\pdm}{\partial_\mu}  % \partial_\mu
\newcommand{\pdn}{\partial_\nu}  % \partial_\nu
\newcommand{\pdup}{\partial^\mu}  % \partial^\mu
\newcommand{\pdupnu}{\partial^\nu}  % \partial^\nu
\newcommand{\Lag}{\mathcal{L}}  % Lagrangian shorthand
\newcommand{\onehalf}{\frac{1}{2}}

% Alternative shorter versions
\newcommand{\dmu}{\partial_\mu}
\newcommand{\dnu}{\partial_\nu}
\newcommand{\dmuup}{\partial^\mu}
\newcommand{\dnuup}{\partial^\nu}

% Even shorter versions (if you prefer)
\newcommand{\dm}{\partial_\mu}
\newcommand{\dn}{\partial_\nu}
\newcommand{\dmuu}{\partial^\mu}
\newcommand{\dnuu}{\partial^\nu}

\title{Personal Solutions to Problem Sheet 1 QFT by David Tong}
\author{Baset}
\date{June 2025}

\begin{document}

\maketitle
\section*{1.}
According to Eq. (2),
\begin{equation}
    \begin{split}
        \frac{\partial y}{\partial t} &= \sqrt{\frac{2}{a}}\sum_{n=1}^\infty \sin(\frac{n\pi x}{a})\dot{q}_n\\
        \frac{\partial y}{\partial x} &= \sqrt{\frac{2}{a}}\sum_{n=1}^\infty \cos(\frac{n\pi x}{a})\frac{n\pi}{a}q_n
    \end{split}
\end{equation}
implying (assuming the integrals are finite and convergent, $m,n > 0$):
\begin{equation}
    \begin{split}
        L &= \int_0^a dx \left[\frac{\sigma}{2}\frac{2}{a}\left(\sum_{n=1}^\infty \sin\left(\frac{n\pi x}{a}\right)\dot{q}_n\right)^2 - \frac{T}{2}\frac{2}{a}\left(\sum_{n=1}^\infty \cos\left(\frac{n\pi x}{a}\right)\frac{n\pi}{a}q_n\right)^2 \right]\\
          &=\frac{1}{a}\int_0^a dx\left[\sigma \sum_{m,n}\sin(\frac{n\pi}{a}x)\sin(\frac{m\pi}{a}x)\dot{q}_n\dot{q}_m-T\sum_{m,n}\cos(\frac{n\pi}{a}x)\cos(\frac{m\pi}{a}x)(\frac{\pi}{a})^2 \, mn\, q_nq_m  \right] \\
          &= \frac{1}{a}\sum_{m,n}\int_0^a dx \left[\sigma \frac{\cos((m-n)\frac{\pi}{a}x)-\cos((m+n)\frac{\pi}{a}x)}{2}\dot{q}_n\dot{q}_m - T \frac{\cos((m-n)\frac{\pi}{a}x)+\cos((m+n)\frac{\pi}{a}x)}{2}\right.\\
          &\left.\quad(\frac{\pi}{a})^2 \, mn\, q_nq_m \right] = \frac{1}{a}\sum_{m,n} \left[ \frac{\sigma}{2}\delta_{m,n}\dot{q}_n\dot{q}_m - (\frac{\pi}{a})^2 \delta_{m,n} \, mn\, q_nq_m \frac{T}{2} \right] \\
          &= \sum_{n=1}^\infty \left[ \frac{\sigma}{2}\dot{q}_n^2 - \frac{T}{2}(\frac{n\pi}{a})^2 q_n^2  \right] = L(q_n,\dot{q}_n)  
        \end{split}
\end{equation}
E-L equations imply:

\begin{equation}
    \begin{split}
       & -\frac{\partial L}{\partial q_n} + \partial_\mu \frac{\partial L}{\partial_\mu(q_n)} = 0 \quad \equiv T(\frac{n\pi}{a})^2q_n + \sigma \ddot{q}_n = 0 \implies \\
       & \ddot{q}_n + \frac{T}{\sigma}(\frac{n\pi}{a})^2 q_n = 0 \implies \omega_n = \sqrt{\frac{T}{\sigma}} (\frac{n\pi}{a})\\
    \end{split}
\end{equation}


\section*{2.}
According to the notation in the notes $\phi'(x) = \phi(y)$, implying
\begin{equation}
    y^\nu = (\Lambda^{-1})_\mu^\nu x^\mu \implies \frac{\partial y^\nu}{\partial x^\mu} = (\Lambda^{-1})_\mu^\nu
\end{equation}
The KG in the notes is $\partial_\mu \partial^\mu \phi(x) + m^2 \phi(x) = 0$, the KG in the new 
coordinate system:
\begin{equation}
    \begin{split}
       \text{KG}': &(\Lambda^{-1})^\mu_\nu \partial_\mu (\Lambda^{-1})_\mu^\nu \partial^\mu \phi(y) + m^2 \phi(y) = 0\\
        & \equiv \partial_\nu \partial^\nu \phi(y) + m^2\phi(y) = 0  
    \end{split}
\end{equation}

\section*{3.}
E-L equations:
\begin{equation}
    \begin{cases}
         \text{Eq1: }& \pdm \frac{\pd \Lag}{\pd (\pdm \psi^*)} - \frac{\pd \Lag}{\pd \psi^*} = 0\\
         \text{Eq2: }& \pdm \frac{\pd \Lag}{\pd (\pdm \psi)} - \frac{\pd \Lag}{\pd \psi} = 0
    \end{cases}
    \implies 
    \begin{cases}
        \text{- }& \pdm \pdup \psi + m^2 \psi + \lambda \psi^*\psi^2 = 0 \\
        \text{- }& \pdm \pdup \psi^* + m^2 \psi^* + \lambda \psi(\psi^*)^2 = 0 
    \end{cases}
\end{equation}
The invariance part and the changed Lagrangian up to the first order:
\begin{equation}
    \begin{split}
    \Lag' & \simeq \pdm (\psi^* -i\alpha\psi^*) \pdup (\psi + i\alpha\psi)-m^2\psi^*\psi -\frac{\lambda}{2}(\psi^*\psi)^2 \\
        & \simeq \pdm\psi^* \pdup\psi -m^2\psi^*\psi -\frac{\lambda}{2}(\psi^*\psi)^2 = \Lag
    \end{split} 
\end{equation}
therefore, the conserved current is:
\begin{equation}
    \begin{split}
    j^\mu = &\frac{\pd \Lag }{\pd (\pdm\phi_a)}X_a(\phi) = i\alpha\psi\pdup\psi^* - i\alpha\psi^*\pdup\psi \\
    & \equiv i(\psi\pdup\psi^* - \psi^*\pdup\psi)
    \end{split}
\end{equation}
We calculate explicitly $\pdm j^\mu$ and use the E-L:
\begin{equation}
    \begin{split}
        \pdm j^\mu & = i\left[ \pdm\psi\pdup\psi^* + \psi\pdm\pdup\psi^* - \pdm\psi^* \pdup\psi - \psi^* \pdm\pdup\psi \right]\\
        & = i\left[ -\psi\left(m^2\psi^* + \lambda\psi (\psi^*)^2 \right) + \psi^*(m^2\psi + \lambda\psi^*\psi^2) \right] \\
        & = 0
    \end{split}
\end{equation}
QED.

\section*{4.}
\begin{equation}
    \Lag = \onehalf\pdm \phi_a \pdup \phi_a - \onehalf m^2 \phi_a\phi_a \implies 
\end{equation}

\begin{equation}
    \begin{split}
        \Lag' & \simeq \onehalf\pdm(\phi_a + \theta\epsilon_{abc}n_b\phi_c)\pdup (\phi_a + \theta\epsilon_{abc}n_b\phi_c)-
        \onehalf m^2 (\phi_a\phi_a - 2\theta \epsilon_{bac}\phi_a\phi_c)\\
        & \simeq \onehalf \{\pdm \phi_a \pdup\phi_a + \theta n_b \left[\epsilon_{abc}\pdm\phi_c \pdup\phi_a 
        + \epsilon_{abc}\pdup\phi_c \pdm\phi_a  \right]  \} - \onehalf m^2 \phi_a\phi_a \\
        & = \onehalf\pdm \phi_a \pdup \phi_a - \onehalf m^2 \phi_a\phi_a = \Lag
    \end{split}
\end{equation}
We used the fact that $\epsilon_{abc}$ tensor is anti-symmetric, but $\pdm\phi_c \pdup\phi_a$ is symmetric w.r.t
changing the indices $a$ and $c$. The same goes for $\phi_a\phi_c$.

Computing the Noether current:
\begin{equation}
    \begin{split}
        j^\mu & = \frac{\pd\Lag}{\pd(\pdm \phi_a)}X_a(\phi) \\
        & = \pdup\phi_a \, \epsilon_{abc}n_b\phi_c \implies \\
        Q & = \int d^3x \, \epsilon_{abc} \dot{\phi_a} n_b \phi_c \\
        & =  n_b \int d^3x \,\epsilon_{bca} \dot{\phi_a} \phi_c \\ 
        & = n_b \int d^3x \, \epsilon_{bac} \dot{\phi_c} \phi_a = n_a \int d^3x \, 
        \epsilon_{abc} \dot{\phi_b} \phi_c 
    \end{split}
\end{equation}
We can choose $n_a = (1,0,0)$, then for each choice:
\begin{equation}
    Q_a = \int d^3x \, 
    \epsilon_{abc} \dot{\phi_b} \phi_c 
\end{equation}
Direct confirmation:
\begin{equation}
    \begin{split}
        \Lag & = \onehalf\pd_t\phi_a\pd_t\phi_a - \onehalf\nabla \phi_a\nabla\phi_a-\onehalf m^2\phi_a\phi_a \implies \\
        \text{E-L for each $a$: } & \pd_t^2 \phi_a - \nabla^2\phi_a + m^2\phi_a = 0 
    \end{split}
\end{equation} 
Hence:
\begin{equation}
    \begin{split}
        \frac{d \, Q_a}{dt} & = \int d^3x \, \epsilon_{abc}\ddot{\phi_a}\phi_c + \int  
        d^3x \, \epsilon_{abc}\dot{\phi_a}\dot{\phi_c}  \\
        & = \int d^3x \, \epsilon_{abc}(\nabla^2 \phi_a - m^2\phi_a)\phi_c = 
         \int d^3x \, \epsilon_{abc} \nabla^2 \phi_a \, \phi_c \\
         & = - \int d^3x \, \epsilon_{abc} \nabla\phi_a.\nabla\phi_c = 0
    \end{split}
\end{equation}
We have used the symmetric, anti-symmetric point several times in the above derivation.


\section*{5.}
- The first result is:
\begin{equation}
    \begin{split}
    \eta_{\sigma\tau}x'^\sigma x'^{\tau} & = \eta_{\sigma\tau}\Lambda_\mu^\sigma \, x^\mu \Lambda^\tau_\nu \, x^\nu = \eta_{\mu\nu}x^\mu x^\nu \implies \\
    \eta_{\mu\nu} & = \eta_{\sigma\tau}\Lambda_\mu^\sigma \Lambda^\tau_\nu
    \end{split}
\end{equation}

For any transformation to be Lorentz, it must satisfy Eq.(16):
\begin{equation}
    \begin{split}
        \eta_{\mu\nu} & = \eta_{\sigma\tau}(\delta^\sigma_\mu + \omega^\sigma_\mu)(\delta^\tau_\nu + \omega^\tau_\nu)\\
        \eta_{\mu\nu} & = \eta_{\mu\nu} + \eta_{\mu\tau} \omega^\tau_\nu + \eta_{\sigma\nu}\omega^\sigma_\mu + \eta_{\sigma\tau}\omega^\sigma_\mu\omega^\tau_\nu \\
        & \simeq \eta_{\mu\nu} + (\omega_{\mu\nu} + \omega_{\nu\mu}) 
    \end{split}
\end{equation}
Hence:
\begin{equation}
    \omega^{\mu\nu} = - \omega^{\nu\mu}
\end{equation}
A pure rotation:
\begin{equation}
    \begin{split}
    R(\theta) & = \begin{bmatrix}
            1 & 0 & 0 & 0 \\
            0 & \cos{\theta} & -\sin{\theta} & 0 \\
            0 & \sin{\theta} & \cos{\theta} & 0 \\ 
            0 & 0 & 0 & -1
                \end{bmatrix} \\
    & \simeq \begin{bmatrix}
            1  & 0 & 0 & 0 \\
            0  & 1 & -\theta & 0\\
            0 & \theta & 1 & 0 \\
            0 & 0 & 0 & 1  
            \end{bmatrix} = [\delta] + \begin{bmatrix}
                0  & 0 & 0 & 0 \\
                0  & 0 & \theta & 0\\
                0 & -\theta & 0 & 0\\
                0 & 0 & 0 & 0
            \end{bmatrix}
    \end{split}
\end{equation}
Anti-symmetric as one can see for $\theta \ll 1$ infinitesimally small.
In cas of the boost:
\begin{equation}
    \begin{split}
        \Lambda(v) & = \begin{bmatrix}
                \gamma & -\gamma v & 0 & 0\\
                -\gamma v & \gamma & 0 & 0 \\
                0 & 0 & 1 & 0 \\
                0 & 0 & 0 & 1
                        \end{bmatrix}
    \end{split}
\end{equation}
$\gamma(v) \simeq 1 + \onehalf v^2 \implies $
\begin{equation}
    \begin{split}
        \Lambda(v) & \simeq \begin{bmatrix}
             1 & - v & 0 & 0\\
                -v & 1 & 0 & 0 \\
                0 & 0 & 1 & 0 \\
                0 & 0 & 0 & 1
                        \end{bmatrix} = [\delta] + \begin{bmatrix}
                            0 & - v & 0 & 0\\
                               -v & 0 & 0 & 0 \\
                               0 & 0 & 0 & 0 \\
                               0 & 0 & 0 & 0
                                       \end{bmatrix}
    \end{split}
\end{equation}
Again anti-symmetric.
\end{document}
