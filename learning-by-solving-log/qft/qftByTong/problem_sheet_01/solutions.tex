\documentclass{article}
\usepackage[utf8]{inputenc}
\usepackage{amsmath}
\usepackage{amssymb}
\usepackage{physics}
\usepackage{gensymb}
\usepackage{esvect}
\usepackage[title]{appendix} % or other options
\usepackage{graphicx} % Required for inserting images

% Custom commands for partial derivatives
\newcommand{\pd}{\partial}  % Short for \partial
\newcommand{\pdm}{\partial_\mu}  % \partial_\mu
\newcommand{\pdn}{\partial_\nu}  % \partial_\nu
\newcommand{\pdup}{\partial^\mu}  % \partial^\mu
\newcommand{\pdupnu}{\partial^\nu}  % \partial^\nu
\newcommand{\Lag}{\mathcal{L}}  % Lagrangian shorthand

% Alternative shorter versions
\newcommand{\dmu}{\partial_\mu}
\newcommand{\dnu}{\partial_\nu}
\newcommand{\dmuup}{\partial^\mu}
\newcommand{\dnuup}{\partial^\nu}

% Even shorter versions (if you prefer)
\newcommand{\dm}{\partial_\mu}
\newcommand{\dn}{\partial_\nu}
\newcommand{\dmuu}{\partial^\mu}
\newcommand{\dnuu}{\partial^\nu}

\title{Personal Solutions to Problem Sheet 1 QFT by David Tong}
\author{Baset}
\date{June 2025}

\begin{document}

\maketitle
\section*{1.}
According to Eq. (2),
\begin{equation}
    \begin{split}
        \frac{\partial y}{\partial t} &= \sqrt{\frac{2}{a}}\sum_{n=1}^\infty \sin(\frac{n\pi x}{a})\dot{q}_n\\
        \frac{\partial y}{\partial x} &= \sqrt{\frac{2}{a}}\sum_{n=1}^\infty \cos(\frac{n\pi x}{a})\frac{n\pi}{a}q_n
    \end{split}
\end{equation}
implying (assuming the integrals are finite and convergent, $m,n > 0$):
\begin{equation}
    \begin{split}
        L &= \int_0^a dx \left[\frac{\sigma}{2}\frac{2}{a}\left(\sum_{n=1}^\infty \sin\left(\frac{n\pi x}{a}\right)\dot{q}_n\right)^2 - \frac{T}{2}\frac{2}{a}\left(\sum_{n=1}^\infty \cos\left(\frac{n\pi x}{a}\right)\frac{n\pi}{a}q_n\right)^2 \right]\\
          &=\frac{1}{a}\int_0^a dx\left[\sigma \sum_{m,n}\sin(\frac{n\pi}{a}x)\sin(\frac{m\pi}{a}x)\dot{q}_n\dot{q}_m-T\sum_{m,n}\cos(\frac{n\pi}{a}x)\cos(\frac{m\pi}{a}x)(\frac{\pi}{a})^2 \, mn\, q_nq_m  \right] \\
          &= \frac{1}{a}\sum_{m,n}\int_0^a dx \left[\sigma \frac{\cos((m-n)\frac{\pi}{a}x)-\cos((m+n)\frac{\pi}{a}x)}{2}\dot{q}_n\dot{q}_m - T \frac{\cos((m-n)\frac{\pi}{a}x)+\cos((m+n)\frac{\pi}{a}x)}{2}\right.\\
          &\left.\quad(\frac{\pi}{a})^2 \, mn\, q_nq_m \right] = \frac{1}{a}\sum_{m,n} \left[ \frac{\sigma}{2}\delta_{m,n}\dot{q}_n\dot{q}_m - (\frac{\pi}{a})^2 \delta_{m,n} \, mn\, q_nq_m \frac{T}{2} \right] \\
          &= \sum_{n=1}^\infty \left[ \frac{\sigma}{2}\dot{q}_n^2 - \frac{T}{2}(\frac{n\pi}{a})^2 q_n^2  \right] = L(q_n,\dot{q}_n)  
        \end{split}
\end{equation}
E-L equations imply:

\begin{equation}
    \begin{split}
       & -\frac{\partial L}{\partial q_n} + \partial_\mu \frac{\partial L}{\partial_\mu(q_n)} = 0 \quad \equiv T(\frac{n\pi}{a})^2q_n + \sigma \ddot{q}_n = 0 \implies \\
       & \ddot{q}_n + \frac{T}{\sigma}(\frac{n\pi}{a})^2 q_n = 0 \implies \omega_n = \sqrt{\frac{T}{\sigma}} (\frac{n\pi}{a})\\
    \end{split}
\end{equation}


\section*{2.}
According to the notation in the notes $\phi'(x) = \phi(y)$, implying
\begin{equation}
    y^\nu = (\Lambda^{-1})_\mu^\nu x^\mu \implies \frac{\partial y^\nu}{\partial x^\mu} = (\Lambda^{-1})_\mu^\nu
\end{equation}
The KG in the notes is $\partial_\mu \partial^\mu \phi(x) + m^2 \phi(x) = 0$, the KG in the new 
coordinate system:
\begin{equation}
    \begin{split}
       \text{KG}': &(\Lambda^{-1})^\mu_\nu \partial_\mu (\Lambda^{-1})_\mu^\nu \partial^\mu \phi(y) + m^2 \phi(y) = 0\\
        & \equiv \partial_\nu \partial^\nu \phi(y) + m^2\phi(y) = 0  
    \end{split}
\end{equation}

\section*{3.}
E-L equations:
\begin{equation}
    \begin{cases}
         \text{Eq1: }& \pdm \frac{\pd \Lag}{\pd (\pdm \psi^*)} - \frac{\pd \Lag}{\pd \psi^*} = 0\\
         \text{Eq2: }& \pdm \frac{\pd \Lag}{\pd (\pdm \psi)} - \frac{\pd \Lag}{\pd \psi} = 0
    \end{cases}
    \implies 
    \begin{cases}
        \text{- }& \pdm \pdup \psi + m^2 \psi + \lambda \psi^*\psi^2 = 0 \\
        \text{- }& \pdm \pdup \psi^* + m^2 \psi^* + \lambda \psi(\psi^*)^2 = 0 
    \end{cases}
\end{equation}
The invariance part:

\end{document}
